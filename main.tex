

%%
%% This is file `sample-sigconf.tex',
%% generated with the docstrip utility.
%%
%% The original source files were:
%%
%% samples.dtx  (with options: `all,proceedings,bibtex,sigconf')
%% 
%% IMPORTANT NOTICE:
%% 
%% For the copyright see the source file.
%% 
%% Any modified versions of this file must be renamed
%% with new filenames distinct from sample-sigconf.tex.
%% 
%% For distribution of the original source see the terms
%% for copying and modification in the file samples.dtx.
%% 
%% This generated file may be distributed as long as the
%% original source files, as listed above, are part of the
%% same distribution. (The sources need not necessarily be
%% in the same archive or directory.)
%%
%%
%% Commands for TeXCount
%TC:macro \cite [option:text,text]
%TC:macro \citep [option:text,text]
%TC:macro \citet [option:text,text]
%TC:envir table 0 1
%TC:envir table* 0 1
%TC:envir tabular [ignore] word
%TC:envir displaymath 0 word
%TC:envir math 0 word
%TC:envir comment 0 0
%%
%% The first command in your LaTeX source must be the \documentclass
%% command.
%%
%% For submission and review of your manuscript please change the
%% command to \documentclass[manuscript, screen, review]{acmart}.
%%
%% When submitting camera ready or to TAPS, please change the command
%% to \documentclass[sigconf]{acmart} or whichever template is required
%% for your publication.
%%
%%
\documentclass[sigconf]{acmart}




\usepackage{amsmath,amssymb,amsfonts}
\usepackage{algorithmic}
\usepackage{graphicx, subcaption, comment}
\usepackage{textcomp}
\usepackage{makecell}
\usepackage{threeparttable}
\usepackage{tablefootnote}

%%
%% \BibTeX command to typeset BibTeX logo in the docs
\AtBeginDocument{%
  \providecommand\BibTeX{{%
    Bib\TeX}}}


\begin{document}




%%
%% If your work has an appendix, this is the place to put it.
\appendix



\section{Proof for the bound of $P_{m=1}^{\:detect,\mathit{diff}}$}
\label{appendix:proof_diff}

For simplicity, in this appendix we denote the redundancy overhead $r_{l,i}$  as $r$, and the number of devices $N_l$ as $N$. 
$r\in (0,1)$, $N\in [2, +\infty)$, $N\in Z$

\begin{equation}
\begin{split}
P_{m=1}^{\:detect,\mathit{diff}} &= P_{m=1}^{\:detect, cen}-P_{m=1}^{\:detect, ind} \\
&= r - 1 + (1-\frac{r}{N-1})^{({N}-1)} \\
\end{split}
\end{equation}
Assume $x=N-1$, then $x\in [1, +\infty)$, $x\in Z$, then:
\begin{equation}
\begin{split}
P_{m=1}^{\:detect,\mathit{diff}} &= r - 1 + (1-\frac{r}{x})^{x} \\
\end{split}
\end{equation}
Assume $g(x,r) = r - 1 + (1-\frac{r}{x})^{x}$, $f(x,r) = (1-\frac{r}{x})^{x}$. Because $\frac{r}{x} \in (0,1)$, then $1-\frac{r}{x} \in (0,1)$, therefore $f(x,r)=(1-\frac{r}{x})^{x} \in (0,1)$.
\begin{equation}
\begin{split}
\ln f(x,r) &= x\; \ln (1-\frac{r}{x}) \\
\intertext{Derive both sides of equations with x:}
\frac{\frac{df(x,r)}{dx}}{f(x,r)} &= \ln (1-\frac{r}{x}) + x\; \frac{r}{x^2} \frac{1}{1-\frac{r}{x}} \\
&= \ln (1-\frac{r}{x}) + \frac{r}{x-r} \\
\end{split}
\end{equation}
\begin{equation}
\label{eq:fandh}
\begin{split}
{\frac{df(x,r)}{dx}} &= f(x,r) \left(\ln (1-\frac{r}{x}) + \frac{r}{x-r}\right) \\
\end{split}
\end{equation}
Assume $h(x,r) = \ln (1-\frac{r}{x}) + \frac{r}{x-r}$
\begin{equation}
\begin{split}
\frac{dh(x,r)}{dx} &= \frac{1}{1-\frac{r}{x}} (-1)(-\frac{r}{x^2}) - \frac{r}{(x-r)^2} \\
%&= \frac{r}{x(x-r)} - \frac{r}{(x-r)^2} \\
%&= \frac{r(x-r) - rx}{x(x-r)^2} \\
%&= \frac{rx-r^2 - rx}{x(x-r)^2} \\
&= \frac{-r^2}{x(x-r)^2} \\
\end{split}
\end{equation}
Clearly, $\frac{dh(x,r)}{dx} < 0$, this indicates $h(x,r)$ monotonically decreas with the increase of $x$, when $x\in [1, +\infty)$, $x\in Z$. Therefore, we know:
\begin{equation}
\begin{split}
h(x,r)_{min} &= h(x,r)_{x\rightarrow +\infty} \\
&= \ln (1^-) + 0^+ \\
&= 0\\
\end{split}
\end{equation}
Thus $h(x,r) > 0$ is always true when $x\in [1, +\infty)$, $x\in Z$, and $r\in (0,1)$.

From Eq. \ref{eq:fandh}:
\begin{equation}
\begin{split}
{\frac{df(x,r)}{dx}} &= f(x,r) \cdot h(x,r) \\
\end{split}
\end{equation}
As it is proven that $h(x,r) > 0$, $f(x,r)=(1-\frac{r}{x})^{x} \in (0,1) > 0$ in the given range of $x$ and $r$, then we have ${\frac{df(x,r)}{dx}} > 0$. This proves that $f(x,r)$ monotonically increases with the increase of $x$. The upper bound of $f(x,r)$ should be:
\begin{equation}
\begin{split}
f(x,r)_{max} &= f(x,r)_{x \rightarrow +\infty} \\
&= (1-\frac{r}{x})^{x} _{x \rightarrow +\infty} \\
\end{split}
\end{equation}
According to the definition of $e$, $e^k = \lim_{n \rightarrow +\infty} (1+\frac{k}{n})^{n}$, then $f(x,r)_{max} = e^{-r}$.
From here, we can conclude that:
\begin{equation}
\begin{split}
P_{m=1}^{\:detect,\mathit{diff}} = g(x,r) &= r - 1 + (1-\frac{r}{x})^{x} \\
&= r - 1 + f(x,r) \\
&< r - 1 + e^{-r} \\
\end{split}
\end{equation}
To find out the trend of $g(x,r)_{max}$ according to the change of $r$, we derive $g(x,r)$ with r:
\begin{equation}
\begin{split}
g(x,r)_{max} &= r - 1 + e^{-r} \\
\frac{dg(x,r)_{max} }{dr} &= 1 - e^{-r} \in (0, 1-\frac{1}{e}) > 0 \\
\end{split}
\end{equation}
This shows that $g(x,r)_{max}$ increases monotonically with the increase of $r$. The largest possible value of $P_{m=1}^{\:detect,\mathit{diff}}$ is $g(x,r)_{max}(r=1^-) = \frac{1}{e} \approx 0.368$. When $r = 10\%$, the upper bound of $P_{m=1}^{\:detect,\mathit{diff}}$ is around $0.005$, and when $r = 20\%$, the upper bound of $P_{m=1}^{\:detect,\mathit{diff}}$ is around $0.019$.


\begin{comment}
\begin{equation}
\label{diff}
\begin{split}
P_{m=1}^{\:detect, diff} &= P_{m=1}^{\:detect, c}-P_{m=1}^{\:detect, d} \\
&= r_{l,i} - 1 + (1-\frac{r}{N-1})^{({N}-1)} \\
&= r - 1 + \sum_{p=0}^{N-1} \binom{N-1}{p} (-\frac{r}{N-1})^p \\
&\text{Pull out the terms of p=0 and p=1:} \\
&= r - 1 + 1 - r  \\
&\;\;\; + \sum_{p=2}^{N-1} \binom{N-1}{p} (- \frac{r}{N-1})^p \\
&= (-r)^2 \frac{1}{(N-1)^2} \frac{(N-1)(N-2)}{2!} \\
%&\;\;\;+ (-\frac{R_{l,i}}{K_{l,i}})^3 \frac{1}{(N-1)^3} \frac{(N-1)(N-2)(N-3)}{3!}\\
&\;\;\; + ... ... \\
&\;\;\;+ (-r)^{N-1} \frac{1}{(N-1)^{N-1}}  \\
&< (-r)^2 \frac{1}{2!} + ... ... + (-r)^{N-1} \frac{1}{(N-1)!} \\
%+ (-\frac{R_{l,i}}{K_{l,i}})^3 \frac{1}{3!} + (-\frac{R_{l,i}}{K_{l,i}})^4 \frac{1}{4!} \\
\end{split}
\end{equation}
With Taylor expansion $e^x=1+\frac{x^1}{1!}+\frac{x^2}{2!}+ ......$, the formula \ref{diff} can be further simplified:
%
%
\begin{equation}
\label{P_diff_results}
\begin{split}
P_{m=1}^{\:detect, diff} &< e^{-r} - ( 1 + \frac{(-r)^1}{1!})- f(N, r) \\
f(N, r) &= (\frac{(-r)^{N}}{N!} + \frac{(-r)^{N+1}}{(N+1)!}+......\;) \\
%&= e^{-\frac{R_{l,i}}{K_{l,i}}} - 1 + \frac{R_{l,i}}{K_{l,i}}\;\;\;\;\;(0<\frac{R_{l,i}}{K_{l,i}}<1) \\
\end{split}
\end{equation}
As we know, $r$ is a rational number between 0 and 1, while $N$ is an integer number greater than $2$. Then, the absolute value of term $\frac{(-r)^{k}}{k!}$ will increase when $k$ increases, and it will be positive if $k$ is an even number, and negative if odd. Therefore, for Eq. \ref{P_diff_results}, if $N$ is an even number, then $f(N, r)$ will be positive, as the first term $\frac{(-r)^{N}}{N!}$ is a positive term with the largest absolute value. 
Then we have:
\begin{equation}
\label{P_diff_Nl_even}
\begin{split}
P_{m=1}^{\:detect, diff} &< e^{-r} - ( 1 + \frac{(-r)^1}{1!}) \\
&=  e^{-r} - 1 + r \\
\end{split}
\end{equation}
Assume $f_{even}(r)=e^{-r} - 1 + r$. To find the range for it, derivative is performed:
\begin{equation}
\label{P_diff_Nl_even_derivative}
\begin{split}
f_{even}^{\;'}(r) &= -e^{-r} + 1 \in (0, \;1-\frac{1}{e}) > 0\\
\end{split}
\end{equation}
The fact that $f_{even}^{\;'}(r)>0$ indicates $f_{even}(r)$ increases monotonically. Thus, the upperbound of $P_{m=1}^{\:detect, diff}$ will be $f_{even}(r=1)=\frac{1}{e}$.


However, if $N$, is an odd number, then:
\begin{equation}
\label{P_diff_Nl_odd}
\begin{split}
f(N, r) &= (\frac{(-r)^{N}}{N!} + \frac{(-r)^{N+1}}{(N+1)!}+......\;) \\
&> \frac{(-r)^{N}}{N!} \\
\text{Therefore, we have}\\
P_{m=1}^{\:detect, diff} &< e^{-r} - ( 1 + \frac{(-r)^1}{1!}) -\frac{(-r)^{N}}{N!} \\
&= e^{-r} - 1 + r - \frac{(-r)^{N}}{N!}
\end{split}
\end{equation}
Assume $f_{odd}(r)=e^{-r} - 1 + r - \frac{(-r)^{N}}{N!}$:
\begin{equation}
\label{P_diff_Nl_odd_derivative}
\begin{split}
f_{odd}^{\;'}(r) &= -e^{-r} + 1 + \frac{(-r)^{N-1}}{(N-1)!} \\
\end{split}
\end{equation}
Becuase the range of $r$ is from 0 to 1, the smallest possible value for $e^{-r}$ is $\frac{1}{e}$. In addition, $N-1$ is an even number. This shows that $f_{odd}^{\;'}(r) > 1-\frac{1}{e} > 0$, it further indicates $f_{odd}(r)$ increases monotonically, and has an upperbound equal to $\frac{1}{e}+\frac{1}{N!}$.
\end{comment}


\end{document}
\endinput
%%
%% End of file `sample-sigconf.tex'.
